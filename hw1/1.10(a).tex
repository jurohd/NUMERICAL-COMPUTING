\documentclass[]{article}
\usepackage{datetime}
\usepackage{color,array,graphics}
\usepackage{enumerate}
\usepackage{tikz}
\usepackage{geometry}
\usetikzlibrary{arrows,automata}
\usepackage{multirow}
\usepackage{graphicx}
\usepackage{listings}
\usepackage{amsmath}
\usepackage{amssymb}
\usepackage{pgfplots}
\usepackage{romannum}

\setlength{\textheight}{9in}
\setlength{\textwidth}{6.5in}
\setlength{\oddsidemargin}{0in}
\setlength{\evensidemargin}{0in}
\voffset0.0in

\def\OR{\vee}
\def\AND{\wedge}
\def\imp{\rightarrow}
\def\math#1{$#1$}
\def\mand#1{$$#1$$}
\def\mld#1{\begin{equation}
#1
\end{equation}}
\def\eqar#1{\begin{eqnarray}
#1
\end{eqnarray}}
\def\eqan#1{\begin{eqnarray*}
#1
\end{eqnarray*}}
\def\cl#1{{\cal #1}}

\DeclareSymbolFont{AMSb}{U}{msb}{m}{n}
\DeclareMathSymbol{\N}{\mathbin}{AMSb}{"4E}
\DeclareMathSymbol{\Z}{\mathbin}{AMSb}{"5A}
\DeclareMathSymbol{\R}{\mathbin}{AMSb}{"52}
\DeclareMathSymbol{\Q}{\mathbin}{AMSb}{"51}
\DeclareMathSymbol{\I}{\mathbin}{AMSb}{"49}
\DeclareMathSymbol{\C}{\mathbin}{AMSb}{"43}

\usepackage{color} %red, green, blue, yellow, cyan, magenta, black, white
\definecolor{mygreen}{RGB}{28,172,0} % color values Red, Green, Blue
\definecolor{mylilas}{RGB}{170,55,241}
\lstset{language=Matlab,%
    %basicstyle=\color{red},
    breaklines=true,%
    morekeywords={matlab2tikz},
    keywordstyle=\color{blue},%
    morekeywords=[2]{1}, keywordstyle=[2]{\color{black}},
    identifierstyle=\color{black},%
    stringstyle=\color{mylilas},
    commentstyle=\color{mygreen},%
    showstringspaces=false,%without this there will be a symbol in the places where there is a space
    numbers=left,%
    numberstyle={\tiny \color{black}},% size of the numbers
    numbersep=9pt, % this defines how far the numbers are from the text
    emph=[1]{for,end,break},emphstyle=[1]\color{red}, %some words to emphasise
    %emph=[2]{word1,word2}, emphstyle=[2]{style},    
}

\begin{document}
\section *{1.10}
(a)
\par $f(x) = \sqrt{1+x^2}$\\\\
\par $f'(x) = \frac{1}{2}(1+x)^{-1/2}\cdot2x=\frac{x}{\sqrt{1+x^2}}$\\\\
\par $f''(x) = \frac{x'\cdot\sqrt{1+x^2}-(\sqrt{1+x^2})'\cdot x}{(\sqrt{1+x^2})^2}=\sqrt{1+x^2}^{-\frac{3}{2}}$\\\\
\par $f'''(x) = -\frac{3x}{(1+x^2)^\frac{5}{2}}$\\\\
\par $f(0)=1 \ \ f'(0)=0 \ \ f''(0)=0 \ \ f'''(0)=0\\\\$
\par By Taylor's Theorem:$$f(x) = f(0)+f'(0)x+\frac{f''(0)}{2}x^2+\frac{f'''(0)}{6}x^3+\cdots \approx f(0) = 1$$
\par Hence $$\displaystyle{\lim_{x\to 0}f(x)}=\frac{\sqrt{1+x^2}-1}{x^2}=\frac{(\sqrt{1+x^2}-1)(\sqrt{1+x^2}+1)}{x^2(\sqrt{1+x^2}+1)}=\frac{x^2}{x^2(\sqrt{1+x^2}+1)}=\frac{1}{\sqrt{1+x^2}+1}=\frac{1}{2}$$\\
(b)
\par Since $f(x)$ is an even function so W.L.O.G we analyze for $x>0$, as we observe the value of $\sqrt{1+x^2}$ is a little larger than 1 for smaller values of x. In floating-point numbers just to the right of $x=1$ we have $x=1+\epsilon$, $x=1+2\epsilon\cdots$ Then due to round-off rule of the system, for $x^2<\epsilon/2$, $1+x^2$ will be round to 1 and that causes $$f(x)=\frac{\sqrt{1}-1}{x^2}=0$$ And for real number $b=\sqrt{\frac{\epsilon}{2}}=10^{-8}$, if $x<b$, we have $f(x)=0$. \\\\Additionally the oscillations in the graph is caused by $\sqrt{1+x^2}$-value passes from one floating point interval (such as $(1,1+\frac{\epsilon}{2})$) to the next one (such as $(1+\frac{\epsilon}{2},1+\frac{3\epsilon}{2})$) and when $x=10^{-8}$,  $f(x)$ is jumping in to a value less than 1. Also notice that in every interval the function could be expressed as $$f(x)=\frac{c\cdot\epsilon}{x^2}, \  \ \exists c\in\mathbb{R_+}$$ so the function produces a hyperbolic curve in every interval.
\newpage
\section *{1.8(a) Extra credit}
\par For $k\geq 1$, since $e^{-k}<1$ we have  $$\frac{e^k}{1+e^k} = \frac{1}{e^{-k}+1} > \frac{1}{1+1} = \frac{1}{2}$$ Also notice that $$\frac{e^k}{1+e^k}<1$$ thus we have partial sum inequality $$\displaystyle{\sum_{k=1}^{1000}\frac{1}{2}}<\displaystyle{\sum_{k=1}^{1000}\frac{e^k}{1+e^k}}<\displaystyle{\sum_{k=1}^{1000}1}$$ since for $k=0$ $$\frac{1}{2}\leq \frac{e^k}{1+e^k}<1$$ then we have $$\displaystyle{\sum_{k=1}^{1000}\frac{1}{2}}+\frac{1}{2}<\displaystyle{\sum_{k=1}^{1000}\frac{e^k}{1+e^k}}+\frac{e^0}{1+e^0}<\displaystyle{\sum_{k=1}^{1000}1}+1$$which is $$\frac{1001}{2}<\displaystyle{\sum_{k=0}^{1000}\frac{e^k}{1+e^k}}<1001$$


\end{document}a